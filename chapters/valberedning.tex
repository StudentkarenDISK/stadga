\clearpage
\section{Valberedning}
\label{sec:valberedning}

	\subsection{Valberedningen}
	\label{subsec:valberedningen}
		Valberedningen utses på Ekonomiskt årsmöte och skall bestå av minst 3 ledamöter varav en är sammankallande. Minst en av ledamöterna skall ej vara medlem av DISKs styrelse. En jämn könsfördelning och representation från olika sektioner skall beaktas. Valberedningen skall senast 1 oktober ledigförklara styrelseposterna genom anslag på DISKs anslagstavlor eller elektronisk motsvarighet.\\ \\
		Valberedningen skall till Valmötet föreslå en ny styrelse innehållande ordförande, vice ordförande, kassör, vice kassör samt fem till sju övriga ordinarie ledamöter och två till fyra suppleanter. Valberedningen skall dessutom föreslå inspektor, revisorer samt revisorsuppleanter. Valberedning skall beakta en jämn könsfördelning inom styrelsen och presidiet.\\ \\
		Kandidaterna till posterna ordförande, vice ordförande, kassör, vice kassör, revisor och revisorsuppleant skall vara personligt nominerade för respektive post medan övriga kandidaturer gäller ordinarie ledamot. Till suppleantposterna skall valberedningen ange den ordning suppleanterna skall tillträda vid bortfall av ordinarie ledamöter.

	\subsection{Valberedningens förslag}
	\label{subsec:valberedningensforslag}
		Valberedningen får inte nominera någon ledamot av valberedningen till posterna ordförande, vice ordförande, kassör eller vice kassör. Ledamot av valberedningen får ej deltaga i något beslut som rör den egna personens kandidatur.\\ \\
		Valberedningens förslag skall bara innehålla valbara personer enligt \ref{subsec:arendenutomforedragningslista}. Valberedningen är skyldig att informera nominerade om styrelsearbetets innebörd.
Nominerade kandidater till posterna ordförande, vice ordförande, samt kassör skall uppmärksammas på denna stadgas \ref{subsec:andrafortroendeuppdrag}.

	\subsection{Entledigande av ledamot av valberedningen}
	\label{subsec:entledigandeavledamotavvalberedningen}
		Ledamot av valberedningen som önskar avsäga sig sitt uppdrag hemställer om detta till styrelsen vilken entledigar denna, samt meddelar årsmöte sitt beslut. Ledamot av valberedningen kan entledigas av årsmöte som tagit ställning till motion eller proposition innehållande hemställan om entledigande och motivering. Sådant beslut skall fattas med minst 2/3 majoritet.

	\subsection{Fyllnadsval till valberedningen}
	\label{subsec:fyllnadsvaltillvalberedningen}
		Fyllnadsval till valberedningen skall göras om antalet medlemmar av valberedningen efter ett entledigande understiger tre, men bör göras oavsett valberedningens storlek. Vid fyllnadsval av ledamot av valberedningen skall styrelsen kalla till Extra årsmöte. Valberedningen skall ledigförklara posten som skall nybesättas samt föreslå valbar person.

	\subsection{Offentliggörande}
	\label{subsec:offentliggorande}
		Valberedningen skall anslå sitt förslag senast sju dagar innan utsatt mötesdatum på DISKs anslagstavlor eller elektronisk motsvarighet.