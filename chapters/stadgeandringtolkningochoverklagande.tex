\clearpage


\section{Stadgeändring, tolkning och överklagande}
\label{sec:stadgeandringtolkningochoverklagande}

	\subsection{Stadgeändring}
	\label{subsec:stadgeandring}
		För beslut om ändring av stadga fordras beslut fattade med minst 2/3 majoritet vid två, på varandra följande, årsmöten. Förändrad stadga skall tillställas inspektor samt sakrevisor. Förslag om ändring kan framläggas som proposition eller motion. Styrelsen äger när den är helt enig rätt att ändra grammatiska fel och stavfel i denna stadga. Sådan ändring skall meddelas årsmöte.

	\subsection{Ikraftträdande}
	\label{subsec:ikrafttradande}
		Stadgeändring träder i kraft när årsmötet så beslutar.

	\subsection{Tolkning}
	\label{subsec:tolkning}
		Uppstår tvist om tolkning av dessa stadgar skall frågan hänskjutas till inspektor och sakrevisor för gemensamt avgörande.

	\subsection{Överklagande}
	\label{subsec:overklagande}
		Har DISKs årsmöte eller styrelse fattat beslut som uppenbarligen strider mot dessa stadgar får beslutet överklagas till Inspektor och sakrevisor, vilka i förening har rätt att häva beslutet. Prövning enligt ovan skall ske endast om det begärts skriftligen av minst 1/10 eller minst 25 av DISKs medlemmar inom tre veckor från och med den dag då beslutet gavs till känna.

