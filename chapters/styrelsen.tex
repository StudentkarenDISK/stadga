\clearpage
\section{Styrelsen}
\label{sec:styrelsen}

        \subsection{DISKs ledning}
        \label{subsec:disksledning}
                Styrelsen är DISKs verkställande organ. Styrelsen utövar den omedelbara ledningen av DISKs verksamhet vilken skall ske i överensstämmelse med dessa stadgar samt i enlighet med beslut fattade av årsmöte. Styrelsen kan utöva årsmötets befogenhet i ärende som oundgängligen inte tål uppskov. Sådant beslut skall genast meddelas inspektor och sakrevisor. I det fall att beslutet kan påverka föreningens ekonomi skall även den ekonomiska revisorn meddelas. Beslutet skall snarast möjligt underställas årsmöte.

        \subsection{Mandatperiod}
        \label{subsec:mandatperiod}
                Styrelsens mandatperiod sammanfaller med verksamhetsår.

        \subsection{Sammansättning}
        \label{subsec:sammansattning}
                Styrelsen består av ordförande, vice ordförande, kassör, vice kassör samt fem till sju övriga ordinarie ledamöter, upp till fyra suppleanter samt en representant från varje sektion. Styrelsen utser inom sig sekreterare och övriga befattningar efter behov. Avgår ledamot före mandattidens utgång inträder suppleant i dennes ställe enligt av valmötet beslutad turordning. Styrelsen kan välja att istället göra fyllnadsval enligt \ref{subsec:fyllnadsval}.

        \subsection{Kallelse}
        \label{subsec:kallelse}
                Kallelse till ordinarie styrelsemöte skall anslås på DISKs anslagstavlor, eller elektronisk motsvarighet, senast fem dagar innan utsatt mötesdatum. Rätt att, personligen eller via ombud, utlysa ordinarie styrelsemöte tillkommer:
                \begin{itemize}
                \setlength{\itemsep}{0.0cm}
                \setlength{\parskip}{0.0cm}
                        \item Ordförande eller vice ordförande
                        \item Minst tre styrelseledamöter i förening
                        \item Inspektor
                        \item Revisorer
                \end{itemize}

                \subsubsection{Kallelse för viss angiven fråga}
                \label{subsec:kallelseforvissangivenfraga}
                        Kallelse till extra styrelsemöte för viss angiven fråga skall anslås på DISKs anslagstavlor eller elektronisk motsvarighet. Rätt att utlysa extra styrelsemöten tillkommer:
                        \begin{itemize}
                        \setlength{\itemsep}{0.0cm}
                        \setlength{\parskip}{0.0cm}
                                \item Ordförande eller vice ordförande
                                \item Minst tre styrelseledamöter i förening
                                \item Inspektor
                                \item Revisorer
                        \end{itemize}

\clearpage

        \subsection{Beslutsmässighet och omröstning}
        \label{subsec:beslutsmassighetochomrostning}
                Styrelsen är beslutsmässig när minst halva antalet röstberättigade styrelsemedlemmar är närvarande. Vid förhinder för ledamot inträder suppleant enligt av årsmötet fastställd turordning. För giltigt beslut krävs enkel majoritet. Vid lika röstetal har ordföranden utslagsröst. Röstning får inte ske genom ombud.\par
                Beslut som ej tål uppskov kan tas per capsulam. Beslut tagna per capsulam skall protokollföras på nästkommande styrelsemöte.

        \subsection{Punkter på ordinarie styrelsemöte}
        \label{subsec:punkterpaordinariestyrelsemote}
                Vid varje styrelsemöte skall följande punkter förekomma på föredragningslistan:
                \begin{enumerate}[a.]
                \setlength{\itemsep}{0.0cm}
                \setlength{\parskip}{0.0cm}
                        \item Mötets öppnande
                        \item Val av mötessekreterare
                        \item Fråga om mötets stadgeenliga utlysande
                        \item Fastställande av föredragningslista
                        \item Godkännande av protokoll från föregående möte och/eller val av justerare
                        \item Beslut tagna av presidiet
                        \item Beslut tagna per capsulam
                        \item Rapporter
                        \item Uppföljningar
                \setcounter{enumi}{24}
                        \item Övriga frågor
                        \item Mötets avslutande
                \end{enumerate}

        \subsection{Andra förtroendeuppdrag}
        \label{subsec:andrafortroendeuppdrag}
                Ordförande, vice ordförande och kassör får under mandatperioden inte samtidigt vara ordförande, vice ordförande, kassör eller sektionsrepresentant för någon sektion eller liknande sammanslutning under DISK, ej heller ordförande, vice ordförande eller kassör i någon förening eller något organ som tillhör Stockholms universitets övriga studentkårer eller SU.

        \subsection{Styrelsens arbetsuppgifter}
        \label{subsec:styrelsensarbetsuppgifter}
                Utöver vad som i övrigt anges i dessa stadgar åligger det styrelsen att:
                \begin{itemize}
                \setlength{\itemsep}{0.0cm}
                \setlength{\parskip}{0.0cm}
                        \item Ansvara för DISKs verksamhet och ekonomi inför årsmötet
                        \item Verkställa av årsmötet fattade beslut. Att bereda ärenden inför årsmöte
                        \item Fungera som samordnare mellan DISKs sektioner

                        \item Ansvara för att DISK är representerat i de beredande och beslutande organ som är relevanta för föreningens ändamål

                        \item Utse funktionärer och tjänstemän samt utfärda instruktioner för dessa samt för sektionerna
                        \item Upprätta inkomstbudget och utgiftsbudget samt verksamhetsplan för innevarande verksamhetsår till Ekonomiska årsmötet
                        \item Upprätta preliminär inkomstbudget och utgiftsbudget samt preliminär verksamhetsplan för kommande verksamhetsår till Valmötet
                        \item Med avseende på avslutat verksamhetsår upprätta och förelägga Ekonomiska årsmötet årsberättelse som skall vara undertecknade av samtliga styrelsemedlemmar
                \end{itemize}

\clearpage
        \subsection{De förtroendevaldas arbetsuppgifter}
        \label{subsec:defortroendevaldasarbetsuppgifter}
                De förtroendevalda har för sina arbetsområden att följa de instruktioner som styrelsen utfärdar.

        \subsection{Närvarorätt, yttranderätt och rösträtt}
        \label{subsec:narvarorattyttranderattochrostratt}
                Vid frånvaro av ordinarie ledamot äger suppleanterna rösträtt i den turordning som fastställts på årsmötet. DISKs revisorer, revisorsuppleanter samt inspektor äger närvarorätt, yttranderätt, och förslagsrätt på styrelsemöten. Styrelsen kan välja att adjungera in ytterligare personer med närvarorätt och/eller yttranderätt.

        \subsection{Styrelsemöten}
        \label{subsec:styrelsemoten}
                Ordinarie styrelsemöte skall hållas, med maximalt tre veckors mellanrum, under perioderna \emph{15 januari – 1 juni} samt \emph{1 september – 15 december}. Ordinarie styrelsemöten får även hållas utanför dessa perioder. Ordinarie styrelsemöten är styrelsemöten som utlysts enligt \ref{subsec:kallelse}, första stycket.\par
                Extra styrelsemöten får hållas när som helst på året.

        \subsection{Presidiet}
        \label{subsec:presidiet}
                DISKs presidium utgörs av ordförande och vice ordförande.\par
                Vid förfall av ordförande skall vice ordförande inträda i dennes ställe. Vid förfall av vice ordförande inträder kassören i presidiet, och styrelsen äger rätt att förordna ytterligare annan styrelseledamot att utöva dessas befogenheter.

        \subsection{Presidiets uppgifter}
        \label{subsec:presidietsuppgifter}
                DISKs presidium utövar ledningen av styrelsen samt representerar DISK. Därutöver åligger det presidiet att:
                \begin{itemize}
                \setlength{\itemsep}{0.0cm}
                \setlength{\parskip}{0.0cm}
                        \item Tillse att styrelsen följer dessa stadgar och av årsmötet fattade beslut
                        \item Ansvara för att utskick till årsmöten och styrelsemöten sker stadgeenligt
                        \item Upprätta föredragningslista och kallelse till årsmöten
                        \item Övervaka verkställandet av styrelsens beslut
                        \item Ansvara för styrelsens sammanställande av årsberättelsen
                        \item Tillsammans med kassörerna ansvara för budgetering
                        \item Tillsammans med kassörerna ansvara för styrelsens sammanställande av en ekonomisk berättelse för innevarande år inför Ekonomiska årsmötet och upprätta en preliminär budget för följande år inför Valmötet
                        \item Ansvara för styrelsens upprättande av verksamhetsberättelse inför Ekonomiskt årsmöte samt preliminär verksamhetsplan inför Valmöte
                \end{itemize}

        \subsection{Presidiebeslut}
        \label{subsec:presidiebeslut}
                Presidiet kan fatta mindre betydelsefulla beslut eller beslut som ej tål uppskov. Presidiet ansvarar för egna fattade beslut och skall skyndsamt rapportera dessa skriftligen till styrelsen, med motivation både till varför presidiebeslut var nödvändigt och till det beslut som togs, dock senast innan nästkommande styrelsemöte.

        \subsection{Ordförandens uppgifter}
        \label{subsec:ordforandensuppgifter}
                Det åligger ordföranden särskilt att:
                \begin{itemize}
                \setlength{\itemsep}{0.0cm}
                \setlength{\parskip}{0.0cm}
                        \item Vara DISKs officiella representant och föra DISKs talan i officiella sammanhang
                        \item Öppna årsmöten och att leda styrelsemöten
                        \item Fördela ärenden inom styrelsen och till sektionerna
                        \item På förekommen anledning utöva annan styrelseledamots befogenhet
                        \item Övervaka efterlevnaden av de instruktioner och föreskrifter styrelsen utfärdat
                        \item Underteckna avtal som DISK ingår
                        \item Övervaka efterlevnaden av dessa
                        \item Se över dessa stadgars funktion
                \end{itemize}

        \subsection{Vice ordföranden}
        \label{subsec:viceordforanden}
                Vice ordföranden skall assistera ordföranden i dennes uppgifter.

        \subsection{Första sammanträdet}
        \label{subsec:forstasammantradet}
                Vid styrelsens första sammanträde skall föredragningslistan förutom ordinarie punkter även avhandla punkten ``Tecknande av föreningens firma''.

        \subsection{Överlåtande av beslutanderätten}
        \label{subsec:overlatandeavbeslutanderatten}
                Styrelsen får överlåta sin beslutanderätt i enskilda ärenden eller i vissa grupper av ärenden till sektion, kommitté eller annat organ eller till enskild medlem eller anställd. Den som fattat beslut med stöd av bemyndigande enligt föregående stycke skall fortlöpande underrätta styrelsen härom.
