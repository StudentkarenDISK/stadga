\clearpage
\section{Årsmöte}
\label{sec:arsmote}

        \subsection{Årsmöte}
        \label{subsec:arsmote}
                Årsmöte är DISKs högsta beslutande organ som för sitt utövande har att följa dessa stadgar. Årsmöte är Ekonomiskt årsmöte (se \ref{subsec:ekonomisktarsmote}), valmöte (se \ref{subsec:valmote}) och extra årsmöte (se \ref{subsec:extraarsmote}).

        \subsection{Kallelse och offentliggörande}
        \label{subsec:kallelseochoffentliggorande}
                Allmän kallelse jämte föredragningslista skall för Ekonomiskt årsmöte och valmöte utgå senast tre veckor innan utsatt mötesdatum. Kallelse jämte föredragningslista till extra årsmöte skall utgå senast åtta dagar innan utsatt mötesdatum. Dessa handlingar skall anslås på DISKs anslagstavlor, eller elektronisk motsvarighet. Separat brevförsänd kallelse skall utgå till DISKs revisor och inspektor.\\ \\

                Möteshandlingar, där ingår verksamhetsplaner, verksamhetsberättelser, budget, propositioner samt motioner, skall utgå senast en vecka innan utsatt mötesdatum. Dessa handlingar ska tillgängliggöras elektroniskt till DISKs medlemmar.\\ \\

                Har förslag väckts om stadgeändring, nedläggning eller sammanslagning av DISK med annan förening eller annan fråga av väsentlig betydelse för DISK eller dess medlemmar skall det anges i kallelsen.

        \subsection{Tidpunkt för årsmöten}
        \label{subsec:tidpunktforarsmoten}
                \begin{itemize}
                \setlength{\itemsep}{0.0cm}
                \setlength{\parskip}{0.0cm}
                        \item Ekonomiskt årsmöte hålles årligen någon gång under februari månad
                        \item Valmöte hålles årligen någon gång under perioden \emph{1 november – 15 december}
                        \item Årsmöte kan endast hållas under perioden \emph{15 januari – 1 juni} och \emph{1 september – 15 december}
                        \item Årsmöte får inte hållas på helgdag eller dag före helgdag, eller med mindre än 30 dagars mellanrum
                \end{itemize}

        \subsection{Uppgifter}
        \label{subsec:uppgifter}
                Åt årsmöte är förbehållet att:
                \begin{itemize}
                \setlength{\itemsep}{0.0cm}
                \setlength{\parskip}{0.0cm}
                        \item Verkställa val av styrelse, valberedning och revisor
                        \item Med minst 2/3 majoritet upplösa styrelsen och utlysa nyval
                        \item Granska de verkställande organens verksamhet och meddela eller vägra ansvarsfrihet
                        \item Besluta i frågor gällande DISKs inkomster och utgifter
                        \item Utfärda instruktioner till de verkställande organen och DISKs befattningshavare
                        \item Välja inspektor och hedersmedlemmar
                        \item Besluta om ändring av dessa stadgar
                        \item Verkställa fyllnadsval av styrelse, valberedning och revisor
                        \item Fastställa medlemsavgifter
                        \item Inrätta stipendier
                        \item Inrätta och disponera fonder
                        \item Uppta och bevilja lån
                        \item Besluta om DISKs upplösning
                        \item Bilda och lägga ner sektioner
                \end{itemize}

\clearpage
        \subsection{Punkter på varje årsmöte}
        \label{subsec:punkterpavarjearsmote}
                Vid varje årsmöte skall följande punkter finnas på föredragningslistan:
                \begin{enumerate}[1.]
                \setlength{\itemsep}{0.0cm}
                \setlength{\parskip}{0.0cm}
                        \item Formalia
                        \begin{enumerate}[a.]
                        \setlength{\itemsep}{0.0cm}
                        \setlength{\parskip}{0.0cm}
                                \item Mötets öppnande
                                \item Val av mötessekreterare
                                \item Val av mötesordförande
                                \item Val av justerare tillika rösträknare
                                \item Fråga om mötets stadgeenliga utlysande
                                \item Justering av röstlängd
                                \item Eventuella adjungeringar
                                \item Fastställande av föredragningslista
                        \end{enumerate}
                        \item Propositioner
                        \item Motioner
                \setcounter{enumi}{6}
                        \item Mötets avslutande
                \end{enumerate}

        \subsection{Justering}
        \label{subsec:justering}
                Mötesprotokollet justeras av mötesordföranden och de på mötet valda justerarna.

        \subsection{Rättigheter på årsmöten}
        \label{subsec:rattigheterpaarsmoten}
                Rösträtten är personlig och får inte utövas via ombud. Personer som inte är medlemmar kan adjungeras med närvarorätt, yttranderätt och/eller förslagsrätt om årsmötet så beslutar.

        \subsection{Beslutsmässighet}
        \label{subsec:beslutsmassighet}
                Årsmöte är beslutsmässigt när minst 25 röstberättigade medlemmar närvarar, varav majoriteten inte sitter i DISKs styrelse.\\ \\
                I det fall att DISK har färre än 100 medlemmar gäller endast kravet på att DISKs styrelse inte skall ha egen majoritet.

        \subsection{Valbarhet}
        \label{subsec:valbarhet}
                Valbar till styrelsen och valberedningen är röstberättigad medlem av DISK enligt kapitel \ref{sec:medlemskap} som också är myndig. Tillsvidareanställda i DISK får inte väljas till något förtroendeuppdrag i föreningen.

        \subsection{Ärenden utom föredragningslista}
        \label{subsec:arendenutomforedragningslista}
                Vid Ekonomiskt årsmöte och valmöte behandlas ärende som inte upptas i föredragningslistan eller är en fråga under punkten ”Under mötet väckta frågor” endast om 3/4 av de närvarande röstberättigande medlemmarna så begär under punkten ”Fastställande av föredragningslista”.

        \subsection{Omröstning}
        \label{subsec:omrostning}
                Beslut fattas med enkel majoritet, om inte annat är stadgat. Vid lika röstetal gäller det förslag som mötesordföranden företräder. Om denne inte är röstberättigad avgörs frågan med lott\-ning. Val avgörs genom relativ majoritet, den som erhållit störst antal avgivna röster blir vald oberoende av hur dessa röster förhåller sig till det totala antalet avgivna röster. Personval mellan två kandidater som utfaller med lika avgörs genom lottning. Om fler än två kandidater finnes, och två eller flera kandidater har fått lika och högst röstetal görs en ny omröstning där den kandidat som erhållit lägst antal röster stryks.\\ \\
                Omröstning och val sker öppet med acklamation eller, om så begärs av en röstberättigad medlem, genom votering. Omröstning vid personval, ansvarsfrihet och entledigande av styrelse\-medlem skall dock företas med slutna sedlar om någon röstberättigad medlem så begär.\\ \\
Röstberättigad medlem måste avstå från sin röst i följande fall:

                \begin{itemize}
                \setlength{\itemsep}{0.0cm}
                \setlength{\parskip}{0.0cm}
                        \item Vid beslut om ansvarsfrihet för förvaltningsåtgärd då denne varit ansvarig
                        \item Vid val av revisorer som skall granska dennes förvaltning
                        \item Vid beslut om avtal mellan medlemmen och DISK
                        \item Eljest i fråga där jäv kan tänkas föreligga
                \end{itemize}

        \subsection{Motioner och propositioner}
        \label{subsec:motionerochpropositioner}
                Till årsmöte äger DISKs medlemmar (se kapitel \ref{sec:medlemskap}) motionsrätt. Motioner skall vara underskrivna av motionären och skall vara styrelsen tillhanda senast 14 dagar innan utsatt mötesdatum.
                Till årsmöten äger DISKs styrelse propositionsrätt.\\ \\
Motioner med styrelsens yrkande, samt propositioner innehållande motivering och hemställan, skall anslås på DISKs anslagstavlor, eller elektronisk motsvarighet, senast sju dagar innan utsatt mötesdatum.

        \subsection{Ekonomiskt årsmöte}
        \label{subsec:ekonomisktarsmote}
                Vid Ekonomiskt årsmöte skall föredragningslistan, utöver vad som sägs i \ref{subsec:punkterpavarjearsmote}, innehålla dessa punkter:
                \begin{enumerate}[1.]
                \setlength{\itemsep}{0.0cm}
                \setlength{\parskip}{0.0cm}
                \setcounter{enumi}{3}
                        \item Dechargeärenden
                        \begin{enumerate}[a.]
                        \setlength{\itemsep}{0.0cm}
                        \setlength{\parskip}{0.0cm}
                                \item Godkännande av verksamhetsberättelse för föregående år
                                \item Godkännande av ekonomisk berättelse för föregående år
                                \item Revisionsberättelser för föregående verksamhetsår
                                \item Fråga om ansvarsfrihet för styrelsen för föregående verksamhetsår
                                \item Beslut om disposition av överskott eller täckande av underskott från föregående år
                                \item Fastställande av verksamhetsårets verksamhetsplan
                                \item Fastställande av verksamhetsårets budget
                                \end{enumerate}
                        \item Valärenden
                        \begin{enumerate}[a.]
                        \setlength{\itemsep}{0.0cm}
                        \setlength{\parskip}{0.0cm}
                                \item Val av valberedning och sammankallande för valberedningen
                                \end{enumerate}
                \setcounter{enumi}{5}
                        \item Under mötet väckta frågor
                        \end{enumerate}

\clearpage
        \subsection{Valmöte}
        \label{subsec:valmote}
                Vid valmöte skall föredragningslistan utöver vad som sägs i \ref{subsec:punkterpavarjearsmote} innehålla följande punkter:
                \begin{enumerate}[1.]
                \setlength{\itemsep}{0.0cm}
                \setlength{\parskip}{0.0cm}
                \setcounter{enumi}{3}
                        \item Dechargeärenden
                        \begin{enumerate}[a.]
                        \setlength{\itemsep}{0.0cm}
                        \setlength{\parskip}{0.0cm}
                                \item Fastställande av preliminär verksamhetsplan för nästa verksamhetsår
                                \item Fastställande av preliminär budget för nästa verksamhetsår
                                \end{enumerate}
                        \item Valärenden
                        \begin{enumerate}[a.]
                        \setlength{\itemsep}{0.0cm}
                        \setlength{\parskip}{0.0cm}
                                \item Val av ordförande för nästa verksamhetsår
                                \item Val av vice ordförande för nästa verksamhetsår
                                \item Val av kassör för nästa verksamhetsår
                                \item Val av vice kassör för nästa verksamhetsår
                                \item Val av övriga ordinarie styrelseledamöter för nästa verksamhetsår
                                \item Val av sektionsrepresentanter till styrelsen för nästa verksamhetsår
                                \item Val av styrelsesuppleanter för nästa verksamhetsår
                                \item Val av ekonomisk revisor samt revisorsuppleant för nästa verksamhetsår
                                \item Val av sakrevisor samt revisorsuppleant för nästa verksamhetsår
                                \item Val eller ratificering av inspektor
                                \end{enumerate}
                \setcounter{enumi}{5}
                        \item Under mötet väckta frågor.
                        \end{enumerate}

        \subsection{Extra årsmöte}
        \label{subsec:extraarsmote}
                Vid extra årsmöten får endast på föredragningslistan upptagna ärenden behandlas.

        \subsection{Rätt att påkalla extra årsmöte}
        \label{subsec:rattattpakallaextraarsmote}
                Rätt att påkalla extra årsmöte tillkommer:
                \begin{itemize}
                \setlength{\itemsep}{0.0cm}
                \setlength{\parskip}{0.0cm}
                        \item Styrelsen
                        \item Inspektor
                        \item Ekonomiska revisorn
                        \item Sakrevisorn
                        \item Minst 25 röstberättigade medlemmar, vilka därvid skall inkomma med en skriftlig begäran och motivering till styrelsen
                \end{itemize}

        Det åligger styrelsen att senast sju dagar efter begäran om extra årsmöte inkommit utlysa mötet. Mötet skall hållas så snart det i enlighet med \ref{subsec:kallelseochoffentliggorande} och \ref{subsec:tidpunktforarsmoten} kan ske.

\clearpage
        \subsection{Entledigande}
        \label{subsec:entledigande}
                Förtroendevald som önskar avsäga sig sitt uppdrag hemställer om detta till styrelsen vilken entledigar denna, samt meddelar årsmötet sitt beslut. Sektionsrepresentant som önskar avsäga sig sitt uppdrag hemställer om detta till styrelsen vilken entledigar denna samt ger sektionen i uppdrag att på sektionsmöte fastställa förslag till ny sektionsrepresentant.\\ \\
                Förtroendevald kan entledigas av årsmöte som tagit ställning till motion eller proposition innehållande hemställan om entledigande och motivering. Sådant beslut skall fattas med minst 2/3 majoritet.

        \subsection{Fyllnadsval}
        \label{subsec:fyllnadsval}
                Vid fyllnadsval av förtroendevald skall valberedningen omgående ledigförklara posten som skall nybesättas. Valberedningen ska även föreslå valbar person. Styrelsen har rätt att, efter valberedningens förslag, välja in ledamot eller suppleant samt meddela årsmötet sitt beslut. Sektionsrepresentant kan fyllnadsväljas efter sektionsmötes förslag. Beslut om fyllnadsval skall fattas av en enig styrelse. Vid fyllnadsval av ordförande, vice ordförande, kassör, vice kassör, revisorer, revisorsuppleanter eller inspektor skall styrelsen skyndsamt kalla till extra årsmöte. Valberedningen skall ledigförklara posten som skall nybesättas samt föreslå valbar person.
